\section{09/09}
We can apply the algorithm given previously to actually \emph{find} an
assignment of the boolean variables which will satisfy at least $\sfrac{7m}{8}$
clauses. Specifically, consider the following algorithm
\begin{algorithm}
    \caption{Randomized Max 3-SAT Algorithm}
    \begin{algorithmic}[1]
        \Function{Solve}{\texttt{formula}}
            \State $\texttt{result} \gets 0$
            \While{$\texttt{result} < \sfrac{7m}{8}$}
                \State $\texttt{bool-vec}, \texttt{result} \gets \Call{Try-Solve}{\texttt{formula}}$
            \EndWhile
            \State \Return \texttt{bool-vec}
        \EndFunction
        \Function{Try-Solve}{\texttt{formula}}
            \State $\texttt{bool-vec} \gets \Call{Sample}{\set{0,1}^n}$
            \State $k \gets$ number of satisfied clauses in \texttt{formula}
            \State \Return $(\texttt{bool-vec}, k)$
        \EndFunction
    \end{algorithmic}
\end{algorithm}
Clearly, the above algorithm will \emph{eventually} return the desired boolean
vector. However, we are concerned with how many iterations this algorithm will
require. To determine this, we must first determine the probability that the
randomly selected boolean vector will satisfy at least $\sfrac{7m}{8}$ clauses.

Let $X$ represent the number of satisfied clauses. Recall that $\expectation{X}
= \sfrac{7m}{8}$. Let us call this value $\mu$. Then
\begin{align*}\mu
    &= \expectation{X}\\
    &= \sum_{x=0}^m x\prob{X = x}\\
    &= \sum_{x < \mu} x\prob{X = x} + \sum_{x \geq \mu} x\prob{X = x}\\
    \shortintertext{Notice that, since $x < \mu$, $x \leq \mu - c$ for some constant $0 < c \leq 1$}
    &\leq \left(\mu - c\right)\sum_{x < \mu}\prob{X = x} + m\sum_{x \geq \mu} \prob{X = x}\\
    \shortintertext{setting $p = \prob{X \geq \mu}$}\\
    &= (\mu - c)(1 - p) + mp
    \shortintertext{Solving for $p$}\\
    p &\geq \frac{c}{c - \mu + m}\\
      &= \bigTh{\frac{1}{m}}
\end{align*}

Before we finish this analysis, we should consider the Geometric Random
Variable.

\subsection{Geometric Random Variable}
\begin{definition}{Geometric Random Variable}{}
    Consider some Bernoulli trial, i.e., an event with exactly two possible
    outcomes, denotes ``success'' and ``failure'', and suppose success occurs
    with probability $p$. The \emph{Geometric Distribution}, denoted $X \sim
    G(p)$, is the number of trials necessary until a success occurs. 
\end{definition}

\begin{lemma}{}{}
    Let $X$ be some discrete random variable that takes integer values. Then
    \[\expectation{X} = \sum_{x = 1} \prob{X \geq x}\]
\end{lemma}

\begin{proof}
    \begin{align*}\expectation{X}
        &= \sum_{x=1}^{\infty}x \prob{X=x}\\
        &= \sum_{x=1}^{\infty}\sum_{i=1}^x \prob{X=x}\\
        &= \sum_{x=1}^{\infty}\sum_{i=x}^{\infty}\prob{X=i}\\
        &= \sum_{x=1}^{\infty}\prob{X \geq x}
    \end{align*}
\end{proof}

\begin{theorem}{Expectation of Geometric Random Variable}{}
    Let $X \sim G(p)$. Then $\expectation{X} = \frac{1}{p}$.
\end{theorem}

\begin{proof}
    Observe that $\prob{X = k} = (1 - p)^{k - 1}p$, hence
    \[\prob{X \geq k} = (1 - p)^{k - 1}\]
    and
    \begin{align*}\expectation{X}
        &= \sum_{k=1}^{\infty} k\prob{X = k}\\
        &= \sum_{k=1}^{\infty} \prob{X \geq k}\\
        &= \sum_{k=1}^{\infty} (1 - p)^{i - 1}\\
        &= \frac{1}{1 - (1 - p)}\\
        &= \frac{1}{p}
    \end{align*}
\end{proof}

Now, returning to our Max 3-SAT algorithm, notice that the probability that it
requires $k$ iterations to output an assignment of at least $\sfrac{7m}{8}$
\True clauses is
\begin{align*}(1 - p)^k
    &\leq \left(1 - \frac{c}{m}\right)^k\hbox{ for some $c > 0$}\\
    &\leq e^{\frac{-ck}{m}}\\
    \shortintertext{setting $k = \sfrac{m}{c}\ln{n}$}
    &= \frac{1}{m}
\end{align*}
Observe that our algorithm is a Las Vegas algorithm. However, there is an
analogous Monte Carlo algorithm:
\begin{algorithm}
    \caption{Randomized Max 3-SAT Algorithm (MC Variant)}
    \begin{algorithmic}[1]
        \Function{Solve}{\texttt{formula}}
            \State $\texttt{result} \gets 0$
            \State $\texttt{bool-vec} \gets \vec{0}$
            \ForRange{$i$}{1}{$\sfrac{m}{c}\ln{n}$}
                \State $\texttt{new-bool-vec}, \texttt{new-result} \gets \Call{Try-Solve}{\texttt{formula}}$
                \If{$\texttt{new-result} > \texttt{result}$}
                    \State $\texttt{result} \gets \texttt{new-result}$
                    \State $\texttt{bool-vec} \gets \texttt{new-bool-vec}$
                \EndIf
            \EndForRange
            \State \Return \texttt{bool-vec}
        \EndFunction
    \end{algorithmic}
\end{algorithm}

\subsection{Binomial Random Variable}
\begin{definition}{Geometric Random Variable}{}
    Consider some Bernoulli trial, i.e., an event with exactly two possible
    outcomes, denotes ``success'' and ``failure'', and suppose success occurs
    with probability $p$. The \emph{Binomial Distribution}, denoted $X \sim
    B(n, p)$, is the number of successes after $n$ trials.
\end{definition}

\begin{theorem}{}{}
    Let $X \sim B(n, p)$. Then $\expectation{X} = np$.
\end{theorem}

\begin{proof}
    Again, \nameref{thm:linexp} is useful here. Let $X_i$ be the indicator
    random variable that is 1 when trial $i$ is a success. Then
    \[\expectation{X_i} = \prob{X_i = 1} = p\]
    hence
    \[\expectation{X} = \sum_{i=1}^n \expectation{X_i} = np\qedhere\]
\end{proof}

\subsection{Randomized Quicksort}
The \emph{Quicksort} algorithm is a computer science staple, one of the most
famous algorithms of the 20th century. A simplified version of the original
implementation is below:

\begin{algorithm}
    \caption{Simplified variant of Quicksort algorithm.}
    \label{alg:quicksort}
    \begin{algorithmic}[1]
        \Function{Quicksort}{\texttt{arr}}
            \State $n \gets \Call{Len}{\texttt{arr}}$
            \If{$n \leq 1}$
                \State \Return \texttt{arr}
            \Else
                \State $\texttt{left}, P, \texttt{right} \gets \Call{Partition}{\texttt{arr}}$
                \State \Return $\Call{Quicksort}{\texttt{left}} + P + \Call{Quicksort}{\texttt{right}}$
            \EndIf
        \EndFunction
        \Function{Partition}{\texttt{arr}}
            \State $\texttt{pivot} \gets \texttt{arr}[0]$
            \State $\texttt{left}, P, \texttt{right} \gets [], [], []$
            \For{$a \in \texttt{arr}}$
                \If{$a < \texttt{pivot}}$
                    \State $\texttt{left}.\Call{Append}{a}$
                \ElsIf{$a = \texttt{pivot}$}
                    \State $P.\Call{Append}{a}$
                \Else
                    \State $\texttt{right}.\Call{Append}{a}$
                \EndIf
            \EndFor
            \State \Return $\texttt{left}, P, \texttt{right}$
        \EndFunction
    \end{algorithmic}
\end{algorithm}

Notice that the worst case of this algorithm is $\bigO{n^2}$.
%TODO demonstrate worst case quicksort
On the other hand, the best case is $\bigO{n\log{n}}$.
%TODO demonstrate best case quicksort
We can consider a randomized variant --- functionally, the same algorithm, but
with the partition scheme changed. Rather than selecting the element with index
0 as the pivot, select a \emph{random} element as the pivot. Let us consider the
number of comparisons in this variant.

\begin{theorem}{}{}
    The above randomized variant of Quicksort has an expected runtime of
    $\bigO{n\log{n}}$.
\end{theorem}

\begin{proof}
    Let $X$ denote the number of comparisons. It should be clear that
    $\prob{X=x}$ is impractical to compute directly. Instead, notice that two
    elements of the array will be compared \emph{at most} once --- when one of
    the two elements is the pivot \emph{end}. 

    Assume the sorted order of the elements is $a_0$, $a_1$, \dots, $a_n$, and
    let $X_{i,j}$ denote the indicator random variable that is 1 when $a_i$ and
    $a_j$ are compared. Notice that if any of $a_{i+1}$, $a_{i + 2}$, \dots,
    $a_{j - 1}$ are chosen, then $a_i$ and $a_j$ will end up in separate
    partitions and never be compared. Additionally, if any values $a_0$, $a_1$,
    \dots, $a_{i - 1}$ or $a_{j + 1}$, $a_{j + 2}$, \dots, $a_{n-1}$ are chosen,
    this will have no effect on whether $a_i$ and $a_j$ are compared.
    
    Thus, the probability that $a_i$ and $a_j$ are compared is simply the
    probability that either value is chosen as the pivot from $a_i$, $a_{i +
    1}$, \dots, $a_j$, which is
    \[\frac{2}{j - i + 1}\]
    Thus, we have
    \begin{align*}\expectation{X}
        &=\sum_{i=0}^{n-1}\sum_{j=i+1}^{n-1}\expectation{X_{i,j}}\\
        &=\sum_{i=0}^{n-1}\sum_{j=i+1}^{n-1}\frac{2}{j - i + 1}\\
        \shortintertext{writing $k = j - i$}\\
        &=\sum_{i=0}^{n-1}\sum_{k=1}^{n - i - 1}\frac{2}{k + 1}\\
        &\leq\sum_{i=0}^{n-1}2H_n\\
        &\leq 2nH_n\\
        &=\bigO{n\log{n}}\qedhere
    \end{align*}
\end{proof}