\section{09/02}
The Ben-Or algorithm can be used to solve Binary Consensus with faults.
\begin{algorithm}
    \caption[\AlgName{Ben-Or}]{Ben-Or Algorithm from view of processor $p$ with initial value $\chi_p$}
    \label{alg:ben-or}
    \begin{algorithmic}[1]
        \Procedure{\nameref*{alg:ben-or}}{}
            \State $\texttt{decided} \gets \False$
            \State $\texttt{round} \gets 1$
            \While{not \texttt{decided}}
                \State $\Call{Broadcast}{1, \texttt{round}, \chi_p}$
                \State wait for $n - f$ messages of type $(1, \texttt{round}, \cdot)$
                \If{there are more than $\sfrac{n}{2}$ messages of type $(1, \texttt{round}, v)$}
                    \State $\Call{Broadcast}{2, \texttt{round}, D, v}$
                \Else
                    \State $\Call{Broadcast}{2, \texttt{round}, U}$
                \EndIf
                \State wait for $n - f$ messages of type $(2, \texttt{round}, \cdot)$
                \State $t \gets$ number of messages of type $(2, \texttt{round}, D, v)$
                \If{$t > 0$}
                    \State $\chi_p \gets v$
                    \If{$t > f$}
                        \State $\texttt{decided} \gets \True$
                        \State $\Call{Broadcast}{1, \texttt{round}, \chi_p}$
                    \EndIf
                \Else \Comment{All messages are of type $(2, \texttt{round}, U)$}
                    \State $\chi_p \gets \Call{Sample}{\set{0,1}}$
                \EndIf
                \State $\texttt{round} \gets \texttt{round} + 1$
            \EndWhile
        \EndProcedure
    \end{algorithmic}
\end{algorithm}
\begin{theorem}{}{}
    The above algorithm is correct.
\end{theorem}

\begin{proof}
    Validity is clear: if all nodes start with $v$, then we skip to 
    %TODO finish proof
\end{proof}

\begin{theorem}{}{}
    The expected number of rounds of \nameref{alg:ben-or} is $2^{n-1}$.
    Additionally, with high probability, the number of rounds is less than
    $2^{n-1}\ln{n}$.
\end{theorem}

\begin{proof}
    If all nodes are of type $(2, \texttt{round}, ?)$, then with probability
    $\sfrac{2}{2^n} = \sfrac{1}{2^{n-1}}$ all nodes will choose the same value.
    Thus, we can upper bound the probability of failure after $k$ rounds by
    \begin{align*}\prob{\text{failure}}
        &\leq \left(1 - \frac{1}{2^{n-1}}\right)^k \leq e^{\frac{-k}{2^{n-1}}}\\
        \shortintertext{setting $k=2^{n-1}\ln{n}$}
        &=\frac{1}{n}
    \end{align*}
    Thus, the probability of success after $2^{n-1}\ln{n}$ rounds is at least $1
    - \sfrac{1}{n}$. Additionally, the expected number of rounds is at most
    $2^{n-1}$.
\end{proof}

\subsection{Graphs}
\begin{problem}{Min-Cut}{}
    Given a graph $G = (V, E)$, find a minimum set of \emph{edges} whose removal
    disconnects the graph.
\end{problem}

\begin{definition}{Graph Cut}{}
    A \emph{cut} is a partition of the vertices into two non-empty subsets, $A$
    and $B$.
\end{definition}

Notice that any edge from $A$ to $B$ crosses the cut, and when all such edges
are removed, the graph becomes disconnected. 

%TODO example figures

In general, there are $\binom{n}{2}$ possible min-cuts.

%TODO draw cycle

\begin{algorithm}
    \caption{Karger's Algorithm}
    \label{alg:karger}
    \begin{algorithmic}[1]
        \Procedure{Karger}{}
            \ForRange{$i$}{1}{$n - 2$}
                \State Choose a random edge $e$ \label{line:kargersample}
                \State Contract edge $e$ \label{line:kargercontraction}
            \EndForRange
            \State Output remaining edges
        \EndProcedure
    \end{algorithmic}
\end{algorithm}

Observe that \cref{line:kargersample} requires $\bigO{1}$ operations and
\cref{line:kargercontraction} requires $\bigO{n}$ operations, for a total of
$\bigO{n^2}$. Additionally, a cut set in the contracted graph is a cut set in
the original graph, thus we can only \emph{increase} the size of the min-cut via
\cref{line:kargercontraction}.

\begin{theorem}{}{}
    \nameref{alg:karger} outputs a min-cut with probability at least
    $\sfrac{1}{n^2}$.
\end{theorem}

\begin{proof}
    Suppose $C$ is some min-cut in $G$ and write $\card{C} = k$. In that case,
    every node must have degree at least $k$, and $m \geq \sfrac{nk}{2}$. 

    Let $E_i$ denote the event that no edge in $C$ is contracted in iteration
    $i$. Notice that the event that $C$ is output is simply
    \begin{align*}\prob{\bigcap E_i}
        &=\prob{E_1}\prob{E_2 \given E_1}\cdots\prob{E_{n-1} \given E_1 \cap E_2 \cap \dots \cap E_{n - 2}}\\
        &=\left(1 - \frac{2}{n}\right)\left(1 - \frac{2}{n-1}\right)\cdots\left(\frac{2}{3}\right)\\
        &=\frac{2}{n(n-1)}\\
        &=\frac{1}{\binom{n}{2}}\\
        &\geq\frac{1}{n^2}
    \end{align*}
\end{proof}

Thus, we can repeat this process $n^2\ln{n}$ times to achieve a high probability
that the algorithm outputs a min-cut.