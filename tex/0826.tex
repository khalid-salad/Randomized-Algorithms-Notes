\section{08/26}
\subsection{Birthday Paradox}
The probability that no two people share a birthday is
\begin{align*}\prob{E_1 \cap E_2 \cap \dots \cap E_n}
    &= \prob{E_1}\prob{E_2 \given E_1}\cdots \prob{E_n \given E_1 \cap E_2 \cap \dots \cap E_{n - 1}}\\
    &= 1 \times \left(1 - \frac{1}{365}\right) \times \left(1 - \frac{2}{365}\right) \times \cdots \times \left(1 - \frac{365 - n + 1}{365}\right)
    \shortintertext{Applying the inequality $1 - x \leq e^{-x}$ when $\abs{x} < 1$}\\
    &\leq 1 \cdot e^{-\frac{1}{365}} \cdot e^{-\frac{2}{365}} \cdots e^{-\frac{n - 1}{365}}\\
    &= e^{-\frac{n(n-1)}{2\cdot365}}
\end{align*}
Setting
\[e^{-\frac{n(n-1)}{2\cdot365}} \leq \frac{1}{2}\]
yields $n \geq 23$. 

\subsection{Randomized Algorithms}
\begin{problem}{Toy Problem}{}
Let $A$ be an array of $n$ numbers with the property that at least one number
occurs at least $\sfrac{n}{2}$ times and the rest are distinct. Our goal is to
find the element with duplicate entries.
\end{problem}

Any deterministic algorithm will require at least $\sfrac{n}{2} + 1$ operations.
However, we can construct a randomized algorithm that only requires
$\bigO{\log{n}}$ operations.

We do this by \emph{random sampling}. Assume that choosing a single element in
$\set{a_1, a_2, \dots, a_n}$ requires $\bigO{1}$ time and that we sample
\emph{with replacement}. We will frequently use sampling with replacement
because it allows the samples to be independent, and therefore simplifies the
analysis. Now, consider the following algorithm:

\begin{algorithm}
    \caption{Randomized Algorithm for Toy Problem}
    \begin{algorithmic}[1]
        \Function{Rand-Duplicate-Toy}{\texttt{arr}}
            \Loop\label{line:toyloop}
                \State $\texttt{sample}_1 \gets \Call{Sample}{\texttt{arr}}$
                \State $\texttt{sample}_2 \gets \Call{Sample}{\texttt{arr}}$
                \If{$\texttt{sample}_1 \neq \texttt{sample}_2$}
                    \State \Return $\texttt{sample}_1$
                \EndIf
            \EndLoop
        \EndFunction
    \end{algorithmic}
\end{algorithm}

This algorithm simply chooses two elements (not necessarily distinct) at random
from the array and repeats until the two elements are identical, after which it
outputs the sampled element as the duplicate.

Let us analyze the probability that this algorithm outputs the duplicate value
after a single iteration. Suppose the two values chosen are $a_i$ and $a_j$.
Then
\begin{align*}\prob{a_i = a_j \text{ and } i \neq j}
    &= \frac{\sfrac{n}{2}}{n} \cdot \frac{\sfrac{n}{2} - 1}{n}\\
    &= \frac{1}{2} \left(\frac{1}{2} - \frac{1}{n}\right)\\
    &= \frac{1}{4} - \frac{1}{2n}\\
    &\geq \frac{3}{2} \hbox{ for $n \geq 5$} 
\end{align*}
What is the probability, then, that our algorithm fails after $k$ iterations of
\cref{line:toyloop}? These events are independent, hence the probability is
bounded above by
\begin{align*}\prob{\text{failure}}
    &\leq \left(1 - \frac{3}{20}\right)^k\\
    &\leq e^{-\frac{3}{20}k}
    \shortintertext{Setting $k = \frac{20}{3}\ln{n}$,}\\
    &= e^{-\ln{n}}\\
    &=\frac{1}{n}
\end{align*}

\subsection{Bayes' Law}
\begin{theorem}{Bayes' Law}{}
    For any events $A$ and $B$,
    \[\prob{A \given B} = \frac{\prob{B \given A} \prob{A}}{\prob{B}}\]
\end{theorem}

\subsection{Concepts in Randomized Algorithms}
\begin{definition}{}{}
    An algorithm is \emph{Monte Carlo} if it has a non-zero probability of
    outputting an incorrect solution. If an algorithm will output the correct
    solution with probability 1, it is \emph{Las Vegas}.
\end{definition}

\begin{definition}{High Probability}{}
    Given an input of size $n$, we say an event occurs \emph{with high
    probability} (whp) if it occurs with probability $1 - \sfrac{1}{n^c}$ for
    some $c > 0$.
\end{definition}

\begin{problem}{Modified Toy Problem}{}
    Let $A$ be an array of $n$ numbers with the property that \emph{either} all
    elements are distinct \emph{or} $\sfrac{n}{2}$ elements are repeated.
\end{problem}

Let us define
\begin{align*}
    E_1 &= \hbox{event that exactly $\sfrac{n}{2}$ elements are distinct}\\
    E_2 &= \hbox{event that all $n$ elements are distinct}
\end{align*}

At first, our ``belief'' in $E_1$ and $E_2$ are approximately the same, i.e., we
have no information to determine which event is more likely. However, let us
assume we have sampled two elements with replacement and determined that they
are not equal (call this event $F$). We can update our ``belief'' via Bayes'
Law.
\begin{align*}\prob{E_1 \given F}
    &= \frac{\prob{F \given E_1} \prob{E_1}}{\prob{F}}\\
    &\leq \frac{\left(1 - \frac{3}{20}\right) \cdot \frac{1}{2}}{\left(1 - \frac{3}{20}\right) \cdot \frac{1}{2} + \frac{1}{2}}\\
    &= \frac{17}{37}
\end{align*}
Thus, we see that the probability has shifted from ``50-50'' to skewing \emph{in
favor} of the belief that $E_2$ has occurred.

\subsection{Law of Total Probability}
\begin{theorem}{Law of Total Probability}{}
    For any event $A$ and any countable partition of the sample space $B_1$,
    $B_2$, \dots, that is, the events $B_i$ are pairwise disjoint and their
    union forms the entire sample space, we have
    \[\prob{A} = \sum_i \prob{A \cap B_i} = \sum_i \prob{A \given B_i} \prob{B_i}\]
\end{theorem}

\subsection{Naive Bayesian Classifier}
A Naive Bayesian Classifier is a classic example of how we can apply
probabilistic techniques and Bayes' Law in practice. Suppose we have a
\emph{feature set} $e$, a vector of boolean $n$ boolean variable. These could
be, for example, whether a certain keyword appears in an email in an attempt to
detect spam. Then, assuming the independence of the boolean variables (this is
why it is referred to as \emph{naive}), we have
\begin{align*}\prob{E \given e = (b_1, b_2, \dots, b_n)}
    &= \frac{\prob{e = (b_1, b_2, \dots, b_n) \given E} \prob{E}}{\prob{e = (b_1, b_2, \dots, b_n)}}\\
    &= \frac{\prob{b_1 \given E} \cdot \prob{b_2 \given E} \cdots \prob{b_n \given E} \cdot \prob{event}}{\prob{e = (b_1, b_2, \dots, b_n)}}
\end{align*}